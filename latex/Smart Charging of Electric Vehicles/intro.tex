\chapter{Introduction}
\label{cha:intro}
The first contains a general introduction to the work. The goals are defined and the modus operandi is explained.

\section{Related work}
\subsection{Paper 1}
\textbf{Quantifying flexibility in EV charging as DR potential: Analysis of two real-world data sets}

(I have marked some sentences in this paper which could be useful for my abstract or introduction. Also, some of the plots are very descriptive and could be useful as well for the presentation slides. Furthermore, at the end of the paper there is a very cool explanation of what makes a session flexible.)

In this paper, their goals are to (1) characterize the EV charging behaviour and collect those behaviours into three clusters (park to charge, charging near home and charging near work), (2) to fit statistical models for the characteristics of each cluster (soujourn times and flexibility), and (3) quantify the maximal aggregate load that could be attained by coordinating connected EV charging at a given time \emph{t}, until time $t+\Delta$.

Our problem is encompassed into a local charge near work environment, so I have extracted the results for this cluster in detail. The charge near work cluster has the following characteristics:

\begin{itemize}
  \item Arrivals: Early morning.
  \item Departures: Late afternoon.
  \item Soujourns: Average around 9h.
  \item Resulting flexibility: Mostly on weekdays and during daytime.
\end{itemize}

In order to compare our density distributions from EnergyVille data to this paper's distributions, keep in mind the following values. Sojourn times [min, max] values are [5.00, 18.52] and idle times values are [0, 15.54]. Fitted distributions are logistic so better to check the rest of parameters directly if simulation is needed.

\subsection{Paper 2}
\textbf{Quantitive analysis of electric vehicle flexibility: A data-driven approach}

(I have marked some sentences in this paper which could be useful for my abstract or introduction. Also, some of the plots are very descriptive and could be useful as well for the presentation slides.)

This paper is mainly an extension of the previous paper. Their goals in this case are to (1) characterize the EV charging behaviour and collect those behaviours into three clusters (park to charge, charging near home and charging near work) (same as previous paper), (2) investigate if the characteristics of the charging sessions of each cluster are sensitive to seasonal changes or weekends, and (3) quantify the flexibility exploitation in terms of duration and amount of deferrable energy, as well as finding out which aspects are more useful for various objectives (i.e. load flattening and load balancing against renewable energy sources).

As mentioned before, our problem is encompassed into a local charge near work environment, so I have focused again on the results for this cluster in detail. Our environment doesn't include transactions on weekends so that information is not relevant to us. The rest of characteristics are detailed below: 

\begin{itemize}
  \item Arrival times are predictable because their interquartile ranges are smaller compared to other clusters. Arrival times (around 6-9am).
  \item Sojourn times are typically less than 9 hours and average idle time is 5h 30min, hence, their charging usually happens throughout the day.
  \item Regarding seasonal effects on transactions: arrival times are earlier by around 1 hour in summer and spring, sojourn times and idle times are not affected by seasonal changes.
\end{itemize}

Due to the fact that the arrivals in the charge near work cluster are usually in the morning, the offered flexibility is exploited to cover the afternoon valley but not the night valley, because there are almost no transactions happening overnight. We should then try to focus on encouraging the exploitation of the flexibility of the transactions with morning arrivals to defer the consumption towards the late-morning, early afternoon.

Finally, just to mention that it is demonstrated that the flexibility utilization is clearly influenced not only by the BAU energy consumptions patterns and arrival times, but also by the renewable generation patterns.

(Maybe worth to add the formula of load balancing.)

\subsection{Paper 3}
\textbf{Charging electric vehicles in the smart grid}

(I have marked some sentences in this paper which could be useful for my abstract or introduction. Also, some of the plots are very descriptive and could be useful as well for the presentation slides.)

The first aim of this paper is to provide a general overview about the EV technology, describing the charging process, the charging modes available and the possibilities of communication for the exchange of control information. After that, they provide two examples of case studies based on a residential distribution feeder: in the first one they use load flattening to avoid peaks, and in the second one they use load balancing against a renewable energy source generator (i.e. wind turbines). Finally, they present the available types of demand response algorithms (e.g. distributed, centralized, etc.).

\subsection{Paper 4}
\textbf{Achieving Scalable Model-Free Demand Response in Charging an Electric Vehicle Fleet with Reinforcement Learning}

In this paper, they present a new way of defining the EV charging process, by means of a Markov Decision Process. They propose a way to represent the state-action space and a batch Reinforcement Learning algorithm (i.e. fitted Q-iteration) to learn the optimal EV charging policy.

They tackle the problem of load flattening in this case. They propose to represent states as tuples of two values, departure time and charging time. Actions are defined as a binary value which represents charging or not charging.

The cost function for each state-action pair is the cost of all the connected EV fleet consumption plus the penalty for unfinished charging (i.e. cars which were not fully charged before departure time, departure time < charging time).

The transition probabilities are unknown in this case and they are needed to solve the Bellman equation, so they use FQI to approximate the optimal Q-value-state-action-pair.

\subsection{Paper 5}
\textbf{Definition and evaluation of model-free coordination of electrical vehicle charging with reinforcement learning}

This paper is an extension of the previous one. They formulate a Markov Decision Process to represent the states of the environment while taking into account the EV charging characteristics (i.e. sojourn, arrival and charging times). They propose a Reinforcement Learning based approach for flattening the load of an EV fleet charging.

They ask 4 research questions:

\begin{itemize}
  \item (\textbf{Q1}) What are appropriate parameter settings of the input training data? When the data time span is set to 3 months, the difference in the improvement beyond 5K sample trajectories per each day of the training data is not significant anymore. 
  \item (\textbf{Q2}) How does the RL policy perform compared to an optimal all-knowing oracle algorithm? The proposed approach normalized cost considering 10 and 50 charging stations is 39\% and 30.4\% better respectively, compared to the business-as-usual scenario. It is at the same time 13\% and 15.6\% worse than the normalized cost of the optimal solution, which has perfect knowledge of the future EV charging sessions (i.e. future arrivals, departure times, energy requirements, etc.).
  \item (\textbf{Q3}) How does that performance vary over time (i.e. from one month to the next) using realistic data? The months with larger flexibility have larger cost reductions, even though the cost gap between the learned and optimal policy is still larger for those same months.
  \item (\textbf{Q4}) Does a learned approach generalize to different group sizes? They trained an agent with experiences from 10 EV charging stations and used that same learned policy to control an increased number of charging stations by up to a factor x10. It was proven that from factor 2 onward the performance is approximately the same, so it generalizes well.
\end{itemize}

%%% Local Variables: 
%%% mode: latex
%%% TeX-master: "thesis"
%%% End: 